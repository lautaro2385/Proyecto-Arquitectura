\chapter{Metodología}

\section{Introducción}
Para cumplir con el objetivo trazado se procede a la generación de un producto de tecnología a través de la metodología de desarrollo en cascada teniendo claro que se debe realizar el proceso de levantamiento de requerimientos, el diseño de la aplicación su posterior codificación, la implementación, verificación y mantenimiento. lo anterior aclarando que se deja abierta la posibilidad de realizar mezclas a futuro con metodologías ágiles que permitan liberar componentes adicionales o mejoras que no requieran el proceso de la metodología cascada en su totalidad.

\section{Proceso}

Siendo congruentes con la afirmación de la introducción se realiza el proceso de desarrollo con la metodología en cascada así:
\\
Inicialmente se realiza una reunión con músicos y posibles contratantes para generar un ambiente de operatividad que sirva como marco para las siguientes fases:\\
\itenm{levantamiento de requerimientos}\\
Se procede a especificar los requerimientos funcionales y no funcionales mínimos para que la aplicación tenga éxito con la ayuda de algunos músicos y clientes futuros de la aplicación y sus ideas frente a la misma\\
\itenm{Diseño}\\
Generación de mockups que correspondan a las necesidades planteadas en los requerimientos obtenidos anteriormente\\
\itenm{Codificación}\\
Construcción del producto tecnológico sobre la tecnología seleccionada con los componentes necesarios para satisfacer la necesidad encontrada.\\
\itenm{Implementación} \\
Puesta en productivo del producto generado\\
\itenm{Verificación}\\
Revisión de los resultados obtenidos selección de mejoras y corrección de errores
\itenm{Mantenimiento}\\
Soporte a procesos realizados mediante la aplicación